\section{Discussion} \label{sec:discussion}
The surveyed papers go into multiple interesting directions tackling time-series forecasting through the use of diffusion denoising processes. Most of the models provide probabilistic forecasting, thereby giving more insights in the uncertainty of the predictions. Notably, the D$^3$VAE \cite{li_generative_2022} and DiffLoad \cite{wang_diffload_2023} papers advance this field by dissecting both epistemic and aleatoric uncertainties within predictions. DiffLoad does so by performing the diffusion process on a latent space variable of the historical data \cite{wang_diffload_2023}. Extending on the encoding of historical data, a progression is observed in the methods, evolving from RNNs to Transformers, and ultimately to S4 layers, with the latter demonstrating superior performance in foundational models like \cite{gu_mamba_2023}. Specifically The SSSD$^\text{S4}$ \cite{alcaraz_diffusion-based_2023} and TSDiff \cite{kollovieh_predict_2023} models experiment with this type of layer. Additionally, TSDiff reveals that unconditional forecasting with guidance can be as good as conditional forecasting and can bring more benefits for other time-series related tasks.
With respect to the type of prediction model used in the reverse process, TimeDiff \cite{shen_non-autoregressive_2023} empirically shows the superiority of data prediction models over noise prediction models for forecasting. This is only a recent revelation and not mentioned in any of the other  Furthermore, TimeDiff makes a good comparison with other state-of-the-art transformer forecasting models and shows the superiority of diffusion models by introducing additional indicative bias in the conditioning module. 
Furthermore, throughout the papers, comparative evaluations utilizing the metrics CRPS$_\text{sum}$ \eqref{eq:crps-sum} or MSE \eqref{eq:mse} on diverse datasets like Electricity \cite{trindade_electricityloaddiagrams20112014_2015}, Traffic \cite{cuturi_pems-sf_2011}, or those from Gluon-TS\footref{fn:gluon-ts}, have made the comparisons more straightforward and accessible.
Lastly, the only implementations that make use of the more generalized diffusion process with the use of SDEs are ScoreGrad \cite{yan_scoregrad_2021} and CSPD \cite{bilos_modeling_2022}. With ScoreGrad going as far as to also implement the ODE method enhancing prediction speeds by up to 4.9 times.
Overall, this literature survey provides an in-depth overview of 11 seminal papers in the realm of diffusion-based time-series forecasting models. It serves as an invaluable starting point for future researchers delving into this domain, offering a comprehensive understanding of the current state-of-the-art methodologies and their evolutionary trajectories.

\section{Future Works} \label{sec:future_works}
Future research directions should include the implementation of diffusion models using the ODE method to enhance prediction speeds without compromising accuracy. The adoption of encoder-decoder frameworks for latent space diffusion, moving beyond mere historical data encoding, is also recommended as this has show to improve prediction qualities \cite{rombach_high-resolution_2022}. Continued investigation into the S4 layers \cite{goel_its_2022} is encouraged, given their promising results in surpassing Transformers in foundation models \cite{gu_mamba_2023}. Overall, the integrating of various approaches, such as TimeDiff \cite{shen_non-autoregressive_2023}, DiffLoad \cite{wang_diffload_2023}, $\text{SSSD}^{\text{S4}}$ \cite{alcaraz_diffusion-based_2023}, TDSTF \cite{chang_tdstf_2023}, and DSPD-GP \cite{bilos_modeling_2022}, would be beneficial. This integration could focus on combining extra conditioning information \ref{sec:timediff}, latent space diffusion \ref{sec:diffload}, the use of SSMs \ref{sec:sssds4}, efficient data representation \ref{sec:tdstf}, and modeling time-series as continuous functions with time-dependent noise \ref{sec:dspd_cspd}. Attention should also be given to long-term multivariate time-series forecasting, decomposing epistemic and aleatoric uncertainty, and evaluating models with both data and noise prediction models to establish when each approach is more advantageous.