The foundational works of diffusion models were established by \textcite{ho_denoising_2020, song_score-based_2021}. These foundations are the unconditional generation of data, which means to create data samples without relying on specific conditions, such as text prompts.
Formally, unconditional generation can be described as viewing the training data ${x}$ as a distribution $q(x)$ from which it is possible to extract samples described by $x \in \mathcal{R}^d$. Now the goal is to approximate this distribution as $p_\theta(x)$ and be able to sample new unseen data from this approximation \cite{luo_understanding_2022}.

Diffusion models approximate the distribution by learning how data can be recovered after it has been diffused to pure noise. The model attempts to transform a Gaussian distribution back into the data distribution as illustrated in \autoref{fig:noise_to_data}. This process enables the model to generate data samples from noise, noise is turned into data that resembles the training dataset.

Where \textcite{ho_denoising_2020} describes the diffusion and denoising processes as discrete steps and \textcite{song_score-based_2021} generalizes these processes to continuous time with the use of stochastic differential equations (SDE). The discrete implementation is described in section \ref{sec:unconditional_ddpm} and the continuous version in section \ref{sec:unconditional_sde}.
However, unconditional generation is not useful if specific data samples are desired. For instance, generating images that contain specifics that are described through text prompts \cite{rombach_high-resolution_2022}. Such conditional generation of data is further explained in section \ref{sec:conditional_diff}

\begin{figure}[b]
\centering
    \noindent\begin{tikzpicture}
      \pgfmathsetmacro\GraphWidth{\textwidth/5} 
      \pgfmathsetmacro\GraphOffset{\textwidth/5 / 30} 
    
      \newcommand{\drawborder}[1]{
        \draw [thick] ([xshift=-1mm,yshift=-1mm]#1.south west) rectangle ([xshift=1mm,yshift=1mm]#1.north east);
      }
        \pgfplotsset{
          myaxis/.style={
            width=\GraphWidth, 
            height=\GraphWidth/2,
            scale only axis,
            axis lines=middle,
            ticks=none,
            axis line style={draw=none},
            xmin=-3, xmax=3,
            ymin=0, ymax=0.45,
          }
        }
    
      % First Graph
      \begin{scope}[shift={(0,0)}, rotate=-90]
          \begin{axis}[name=plot1, myaxis]
          \addplot [smooth,thick,domain=-3:3, blue!40] {(gauss(x, 0, 0.25) + gauss(x, 2, 0.3) + gauss(x, -1.8, 0.2)) / 5};
          \end{axis}
          \drawborder{plot1}
          \node [above] at ([xshift=-1mm]plot1.west) {Data};
      \end{scope}
    
      % Second Graph
      \begin{scope}[shift={(\GraphOffset,0)}, rotate=-90]
          \begin{axis}[name=plot2, myaxis]
          \addplot [smooth,thick,domain=-3:3, blue!40] {(gauss(x, 0, 0.4) + gauss(x, 2, 0.5) + gauss(x, -1.5, 0.35)) / 5};
          \end{axis}
          \drawborder{plot2}
      \end{scope}
    
      % % Third Graph
      \begin{scope}[shift={(\GraphOffset*2,0)}, rotate=-90]
          \begin{axis}[name=plot3, myaxis]
          \addplot [smooth,thick,domain=-3:3, blue!40] {(gauss(x, 0, 0.6) + gauss(x, 1.5, 0.5) + gauss(x, -1.3, 0.5)) / 5};
          \end{axis}
          \drawborder{plot3}
      \end{scope}
    
      % % Fourth Graph
        \begin{scope}[shift={(\GraphOffset*3,0)}, rotate=-90]
          \begin{axis}[name=plot4, myaxis]
          \addplot [smooth,thick,domain=-3:3, blue!40] {(gauss(x, 0, 0.8) + gauss(x, 0.7, 0.5) + gauss(x, -0.5, 0.4)) / 5};
          \end{axis}
          \drawborder{plot4}
      \end{scope}

      % % Fourth Graph
        \begin{scope}[shift={(\GraphOffset*4,0)}, rotate=-90]
          \begin{axis}[name=plot5, myaxis]
            \addplot [smooth,thick,domain=-3:3, blue!40] {gauss(x, 0, 1)};
        \end{axis}
          \drawborder{plot5}
          \node [above] at ([xshift=-1mm]plot5.west) {Noise};
      \end{scope}

    
      \draw[thick,->] ([yshift=-4mm]plot5.east) -- node[below] {Denoising / Reverse Process} ([yshift=-5mm]plot1.east);
      
    \end{tikzpicture}
    \caption{Probability Distribution Evolution: From Noise to Data} \label{fig:noise_to_data}
\end{figure}

\subsection{Denoising Diffusion Probabilistic Model (DDPM) (2020) \cite{ho_denoising_2020}} \label{sec:unconditional_ddpm} 

\begin{figure}[ht]
\centering
\begin{tikzpicture}[
    auto,
    >={Latex[width=2mm,length=2mm]},
    data/.style={rectangle, draw=black, fill=green!20, align=center, rounded corners, minimum width=1.5cm, minimum height=1cm},
    every node/.style={font=\sffamily},
    node distance=2cm % default distance, but you can customize each one
]

% Nodes
\node[data] (x0) {$x^0$};
\node (dots1) [right=1cm of x0] {$\cdots$}; % shorter arrow
\node[data] (xk-1) [right=1cm of dots1] {$x^{k-1}$}; % shorter arrow, was xk
\node[data] (xk) [right=4cm of xk-1] {$x^k$}; % longer arrow, was xk-1
\node (dots2) [right=1cm of xk] {$\cdots$}; % shorter arrow
\node[data] (xK) [right=1cm of dots2] {$x^K$};

% Forward process arrows
\draw[->] ([yshift=10pt] x0.east) -- ([yshift=10pt] dots1.west);
\draw[->] ([yshift=10pt] dots1.east) -- ([yshift=10pt] xk-1.west); % was xk
\draw[->] ([yshift=10pt] xk-1.east) -- node[above] {Forward: $q(x^k | x^{k-1})$} ([yshift=10pt] xk.west); % was xk-1
\draw[->] ([yshift=10pt] xk.east) -- ([yshift=10pt] dots2.west); % was xk-1
\draw[->] ([yshift=10pt] dots2.east) --([yshift=10pt] xK.west);

% Reverse process arrows
\draw[->] ([yshift=-10pt] xK.west) -- ([yshift=-10pt] dots2.east);
\draw[->] ([yshift=-10pt] dots2.west) --  ([yshift=-10pt] xk.east); % was xk-1
\draw[->] ([yshift=-10pt] xk.west) -- node[below] {Reverse: $p_\theta(x^{k-1} | x^k)$} ([yshift=-10pt] xk-1.east); % was xk
\draw[->] ([yshift=-10pt] xk-1.west) -- ([yshift=-10pt] dots1.east); % was xk
\draw[->] ([yshift=-10pt] dots1.west) -- ([yshift=-10pt] x0.east);

\end{tikzpicture}
\caption{Illustration of the Denoising Diffusion Probabilistic Model process}
\label{fig:ddpm_forward_reverse_process}
\end{figure}

Core to these models are two pivotal processes: the forward diffusion and the reverse denoising process.

The forward process transforms an input $x^0$ into a noise vector $x^K$ through a Markov Chain, progressively introducing noise over $K$ steps as can be seen in \autoref{fig:ddpm_forward_reverse_process}. The transformation is described as:
\begin{equation} \label{eq:diff_forward_process}
    q(x^{1:K} | x^0) = \prod_{k=1}^{K} q(x^k \mid x^{k-1}) \text{ where } q(x^k|x^{k-1}) = \mathcal{N}(x^k; \sqrt{1 - \beta_k}x^{k-1}, \beta_k \textbf{I}) 
\end{equation}
where $\beta_k \in [0, 1]$ dictates the variance of the noise introduced at each stage. Obtaining $x^k$ is achieved through:
\begin{equation} 
   q(x^k|x^0) = \mathcal{N}(x^k; \sqrt{\alpha_k}x^0, (1- \alpha_k) \mathbf{I}) 
\end{equation}
where $\alpha_k = \prod_{i=1}^{k}(1 - \beta_i)$ and can be also represented in a closed form as:
\begin{equation} \label{eq:ddpm_get_xk}
   x^k(x^0, \epsilon) = \sqrt{\alpha_k}x^0 + \sqrt{1 - \alpha_k}\epsilon \quad \text{where } \epsilon \sim \mathcal{N}(0, \mathbf{I})
\end{equation}

The reverse process transforms the noised data $x^k$ back to its input $x^0$. It is defined by the following Markov chain with $x^K \sim \mathcal{N} (0, \textbf{I})$:
\begin{equation} \label{eq:diff_reverse_process}
     p_\theta(x^{0:K}) = p(x^K)\prod_{k=1}^{K} p_\theta(x^{k-1} \mid x^k) \text{ with } p_\theta(x^{k-1} | x^k) = \mathcal{N} (x^{k-1}; \mu_\theta(x^k, k), \sigma^2_k \mathbf{I})    
\end{equation}
In this backwards process $\mu_\theta(x^k, k)$ is modeled using a neural network and where
\begin{equation} 
    \sigma^2_k = 
    \begin{cases} 
        \beta_k & \text{optimal for } x^0 \sim N(0, \mathbf{I}) \\
        \tilde{\beta}_k = \frac{1 - \alpha_{k-1}}{1 - \alpha_k} \beta_k & \text{optimal for } x^0 \text{ deterministically set to one point.}
    \end{cases} \label{eq:diff_sigma_squared}
\end{equation}
The choices of $\sigma^2_k$ correspond to the upper and lower bounds on reverse process entropy for data with coordinatewise unit variance \cite{sohl-dickstein_deep_2015}.

Training the diffusion model involves minimizing a KL-divergence loss:
\begin{equation} 
    \mathcal{L}_k = D_{KL} \left( q(x^{k-1}|x^k) || p_\theta(x^{k-1}|x^k) \right) \label{eq:diff_train_obj_1}
\end{equation}
For more stability in training and since $x^0$ is known, the forward process $q(x^{k-1}|x^k)$ is replaced by:
\begin{align}
    q(x^{k-1}|x^k, x^0) &= \mathcal{N} (x^{k-1}; \tilde{\mu}_k(x^k, x^0), \tilde{\beta}_k(x^k, k)) \\
    \text{where } \tilde{\mu}_k(x^k, x^0) &= \frac{\sqrt{\alpha_{k-1}}\beta_k}{1 - \alpha_t}x^0 + \frac{\sqrt{(1 - \beta_k)}(1 - \alpha_{k-1})}{1 - \alpha_k}x^k \\
    \text{and } \tilde{\beta}_k &= \frac{1 - \alpha_{k-1}}{1 - \alpha_k}\beta_k 
\end{align}
This modification allows to reparameterize the training objective from \autoref{eq:diff_train_obj_1} to:
\begin{equation} 
    \mathcal{L}_k = \frac{1}{2\sigma^2_k} || \tilde{\mu}_k(x^k, x^0) - \mu_\theta(x^k, k)||^2 
\end{equation}
Where $x^k$ is obtained through \autoref{eq:ddpm_get_xk}. From here, $\mu_\theta(x^k, k)$ can be expressed either by a data prediction model $x_\theta(x^k, k)$ or a noise prediction model $\epsilon_\theta(x^k, k)$ \cite{ho_denoising_2020, benny_dynamic_2022, chang_design_2023}. For image generation, \textcite{ho_denoising_2020} found that the latter performs better. Using a noise prediction model with noise loss function:
\begin{equation} \label{eq:ddpm_noise_prediction}
        \mu_\epsilon(\epsilon_\theta) = \frac{1}{\sqrt{1 - \beta_k}}\left(x^k - \frac{\beta_k}{\sqrt{1 - \alpha_k}} \epsilon_\theta(x^k, k) \right) \text{ with } \mathcal{L}_{\epsilon_\theta} = \mathbb{E}_{k, x^0, \epsilon} \left[|| \epsilon - \epsilon_\theta(x^k, k)||^2 \right]
\end{equation}
Alternatively, using the data prediction model and its data loss function:
\begin{equation} \label{eq:ddpm_data_prediction}
    \mu_x(x_\theta) = \frac{\sqrt{1 - \beta_k}(1 - \alpha_{k-1})}{1 - \alpha_k} x^k + \frac{\sqrt{\alpha_{k-1}}\beta_k}{1 - \alpha_k} x_\theta(x^k, k) \text{ with } \mathcal{L}_{x_\theta} = \mathbb{E}_{k, x^0, \epsilon} \left[|| x^0 - x_\theta(x^k, k)||^2 \right]
\end{equation}
For an elaborate derivation of all of the equations mentioned above, take a look at the paper by \textcite{luo_understanding_2022}. The actual sampling of $x^{k-1}$ is shown in \autoref{fig:prediction_models} where $\sigma_k$ is defined as in \autoref{eq:diff_sigma_squared} and $\mathbf{z} \sim \mathcal{N}(0, \mathbf{I})$ Furthermore, \textcite{benny_dynamic_2022} suggest that the combination of the noise and data methods can enhance performance.

\begin{figure}[ht]
\centering
\begin{tikzpicture}[
    auto,
    >={Latex[width=2mm,length=2mm]},
    data/.style={rectangle, draw=black, fill=green!20, align=center, rounded corners, minimum width=1.5cm, minimum height=1cm},
    model/.style={rectangle, draw=black, fill=red!20, align=center, rounded corners, minimum width=2.5cm, minimum height=1cm},
    every node/.style={font=\sffamily},
    node distance=1.5cm % default distance, but you can customize each one
]

% Nodes
\node (dots1) [right=1cm of x0] {$\cdots$};
\node[data] (xk-1) [right=1cm of dots1] {$x^{k-1}$}; 
\node[data] (xk) [right=9cm of xk-1] {$x^k$}; 
\node (dots2) [right=1cm of xk] {$\cdots$};

% Model node with the equations
\node[model] (nn) [below=25pt of $(xk-1)!0.5!(xk)$] {
    % $
    % \mu_\theta(x^k, k) = 
    % \begin{cases}
    %         \mu_\epsilon(\epsilon_\theta), & \text{(Noise model) \eqref{eq:ddpm_noise_prediction}} \\
    %         \mu_x(x_\theta), & \text{(Data model) \eqref{eq:ddpm_data_prediction}}
    % \end{cases}
    % $
    % $x^{k-1} = \mu_\theta(x^k, k) + \sigma_k \mathbf{z}$
    $x^{k-1} = \mu_\theta(x^k, k) 
    \begin{cases}
            \mu_\epsilon(\epsilon_\theta) & \eqref{eq:ddpm_noise_prediction} \\
            \mu_x(x_\theta) & \eqref{eq:ddpm_data_prediction}
    \end{cases}
    + \sigma_k \mathbf{z}$
};

% \node[above=0.75cm of nn] (equation_q) {$q(x^{k-1}|x^k, x^0) = \mathcal{N} (x^{k-1}; \tilde{\mu}_k(x^k, x^0), \tilde{\beta}_k(x^k, k))$};
\node[above=3pt of nn] (equation_p) {$p_\theta(x^{k-1}|x^k) = \mathcal{N} (x^{k-1}; \mu_\theta(x^k, k), \sigma^2_k \mathbf{I})$};
% \node[above=5pt of nn] (equation_p) {$x^{k-1} = \mu_\theta(x^k, k) + \sigma_k \mathbf{z}$};

% Forward process arrows
\draw[->] ([yshift=10pt] dots1.east) -- ([yshift=10pt] xk-1.west);
\draw[->] ([yshift=10pt] xk-1.east) -- ([yshift=10pt] xk.west); 
\draw[->] ([yshift=10pt] xk.east) -- ([yshift=10pt] dots2.west);

% Reverse process arrows
\draw[->] ([yshift=-10pt] dots2.west) --  ([yshift=-10pt] xk.east); 
\draw[->] ([yshift=-10pt] xk-1.west) -- ([yshift=-10pt] dots1.east); 

% \draw[->] (xk) -- ([xshift=50pt] nn.north); 
% \draw[->] ([xshift=-50pt] nn.north) -- (xk-1); 
\draw[->] ([yshift=-10pt] xk.west) -- ([yshift=0pt] nn.east); 
\draw[->] ([yshift=0pt] nn.west) -- ([yshift=-10pt] xk-1.east);

\end{tikzpicture}
\caption{Illustration for prediction models of the reverse process.}
\label{fig:prediction_models}
\end{figure}
\subsection{Score-based generative modeling through SDE (2021) \cite{song_score-based_2021}} \label{sec:unconditional_sde}

\begin{figure}[b]
\centering
\begin{tikzpicture}[
    auto,
    >={Latex[width=2mm,length=2mm]},
    data/.style={rectangle, draw=black, fill=green!20, align=center, rounded corners, minimum width=1.5cm, minimum height=1cm},
    model/.style={rectangle, draw=black, fill=red!20, align=center, rounded corners, minimum width=2.5cm, minimum height=1cm},
    every node/.style={font=\sffamily},
    node distance=1.5cm % default distance, but you can customize each one
]

% Nodes
\node[data] (x0) {\(x(0)\)};
\node[data] (xK) [right=12cm of x0] {\(x(K)\)};


% Forward process arrows
\draw[->] ([yshift=10pt] x0.east) -- node[above] {Forward SDE: $\md x = f(x, k)\md k + g(k)\md \mathbf{w}$} ([yshift=10pt] xK.west);

% Reverse process arrows
% \draw[->] ([yshift=-10pt] xK.west) -- node[below] {Reverse SDE: $\md x = \left[ f(x, k) - g(k)^2 \nabla_x \log{p_k(x)} \right] \md k + g(k)d\bar{\mathbf{w}}$} ([yshift=-10pt] x0.east);
\draw[->] ([yshift=-10pt] xK.west) -- node[below, align=center] {Reverse SDE: $\md x = \left[ f(x, k) - g(k)^2 \nabla_x \log{p_k(x)} \right] \md k + g(k)d\bar{\mathbf{w}}$ \\ Probability Flow ODE: $\md x = \left[ f(x, k) - \frac{1}{2} g(k)^2 \nabla_x \log p_k(x)\right] \md k$} ([yshift=-10pt] x0.east);

\end{tikzpicture}
\caption{Illustration for prediction models of the reverse process.}
\label{fig:score-based_diffusion_sde_process}
\end{figure}

DDPMs operate in a discrete space, with discrete diffusion and denoising steps. With the use of stochastic differential equations (SDE), this process can be described in continuous time as shown in \autoref{fig:score-based_diffusion_sde_process}. This can be thought of as a generalization of the DDPM, as that is a discrete form of SDE \cite{song_score-based_2021}. 
Keep in mind that even though the score-based generative modeling with SDEs is described in continuous time, the solutions are still solved iteratively \cite{song_score-based_2021}.

The process can be formalized by letting $\mathbf{w}$ and $\bar{\mathbf{w}}$ be a standard Wiener process and its time-reverse version, respectively, and consider a continuous diffusion time $k \in [0, K]$. The forward diffusion process is described with an SDE as:
\begin{equation}
    \md x = f(x, k)\md k + g(k)\md \mathbf{w}
\end{equation}
where $g(k)\md \mathbf{w}$ is the stochastic diffusion and $f(x,k)\md k$ is the deterministic drift. 
On the other hand, the reverse denoising process is described with a time-reverse SDE \cite{anderson_reverse-time_1982}:
\begin{equation} \label{eq:sde_reverse_process}
    \md x = \left[ f(x, k) - g(k)^2 \nabla_x \log{p_k(x)} \right] \md k + g(k)d\bar{\mathbf{w}}
\end{equation}
where the score of the marginal distribution $\nabla_x \log{p_k(x)}$ is to be estimated. It is estimated with a time-dependent score-based model $s_\theta(x, k)$ using the score objective function

\begin{equation} \label{eq:sde_train_obj}
    \mathcal{L}_{s_\theta} = \mathbb{E}_{k, x(0), x(k)} \left[\left|\left| \nabla_{x(k)} \log{p_{0k}(x(k) | x(0))} - s_\theta(x(k), k)\right|\right|^2 \right]    
\end{equation}

Using this objective function the score-based model $s_\theta(x, k)$ can be trained on samples with score matching \cite{song_score-based_2021, hyvarinen_estimation_2005, song_sliced_2020, pang_efficient_2020, holzschuh_score_2023}.

The above formula is a general representation of a SDE, and now to be more specificly tailored to the diffusion process there are multiple methods as described in \textcite{song_score-based_2021}, of which 2 are described as following. 
The DDPM process can also be generalized to a SDE and is termed the variance preserving (VP) SDE:
\begin{equation} \label{eq:sde_vp}
    \md x = - \frac{1}{2}\beta(k)x \md k + \sqrt{\beta(k)}\md \mathbf{w}
\end{equation}
where $\beta(\cdot)$ is a continuous function and $\beta(\frac{k}{K}) = K(1 - \beta_k)$ from \eqref{eq:diff_forward_process} as $K \rightarrow \infty$. Based on the VP SDE, Song et al. \cite{song_score-based_2021} has designed the sub-VP SDE, specialized for likelihoods:
\begin{equation} \label{eq:sde_sub-vp}
     \md x = - \frac{1}{2}\beta(k)x \mathrm{d}k + \sqrt{\beta(k) \left(1 - \me^{-2 \int_0^k \beta(s)ds}  \right) }\md \mathbf{w}
\end{equation}

Song et al. \cite{song_score-based_2021} demonstrated that from each SDE, one can derive an ordinary differential equation (ODE) with the same marginal distributions. This associated ODE of an SDE is called the \textit{probability flow} ODE. Applying this concept to the reverse-SDE \eqref{eq:sde_reverse_process} gives the subsequent probability flow ODE:
\begin{equation} \label{eq:sde_probability_flow}
\md x = \left[ f(x, k) - \frac{1}{2} g(k)^2 \nabla_x \log p_k(x)\right] \md k 
\end{equation}
When the $\nabla_x \log p_k(x)$ in the probability flow is replaced by its approximation $s_\theta(x, k)$, it becomes a special case of neural ODE \cite{chen_neural_2018}. Specifically, it resembles continuous normalizing flows because it transforms data distributions to prior noise distributions and it is fully invertible \cite{grathwohl_ffjord_2018}. Furthermore, this allows exact log-likelihood computation and can be trained via maximum likelihood using standard methods \cite{chen_neural_2018}. 

% \begin{figure}[ht]
% \centering
% \begin{tikzpicture}[
%     auto,
%     >={Latex[width=2mm,length=2mm]},
%     data/.style={rectangle, draw=black, fill=green!20, align=center, rounded corners, minimum width=1.5cm, minimum height=1cm},
%     model/.style={rectangle, draw=black, fill=red!20, align=center, rounded corners, minimum width=2.5cm, minimum height=1cm},
%     every node/.style={font=\sffamily},
%     node distance=1.5cm % default distance, but you can customize each one
% ]

% % Nodes
% \node[data] (x0) {\(x(0)\)};
% \node[data] (xK) [right=12cm of x0] {\(x(K)\)};

% % Model node with the equations
% \node[model] (nn) [below=of $(x0)!0.5!(xK)$] {
%     $s_\theta(x^k, k)  \eqref{eq:sde_train_obj}$
% };

% % Forward process arrows
% \draw[->] ([yshift=10pt] x0.east) -- node[above] {Forward SDE: $\md x = f(x, k)\md k + g(k)\md \mathbf{w}$} ([yshift=10pt] xK.west);

% % Reverse process arrows
% % \draw[->] ([yshift=-10pt] xK.west) -- node[below] {Reverse SDE: $\md x = \left[ f(x, k) - g(k)^2 \nabla_x \log{p_k(x)} \right] \md k + g(k)d\bar{\mathbf{w}}$} ([yshift=-10pt] x0.east);
% % \draw[->] ([yshift=-10pt] xK.west) -- node[below, align=center] {Reverse SDE: $\md x = \left[ f(x, k) - g(k)^2 \nabla_x \log{p_k(x)} \right] \md k + g(k)d\bar{\mathbf{w}}$ \\ Probability Flow ODE: $\md x = \left[ f(x, k) - \frac{1}{2} g(k)^2 \nabla_x \log p_k(x)\right] \md k$} ([yshift=-10pt] x0.east);

% \draw[->] (xK) -- ([xshift=10pt] nn.north);
% \draw[->] ([xshift=-10pt] nn.north) -- (x0);

% \end{tikzpicture}
% \caption{Illustration for prediction models of the reverse process.}
% \label{fig:score-based_diffusion_sde_process}
% \end{figure}

