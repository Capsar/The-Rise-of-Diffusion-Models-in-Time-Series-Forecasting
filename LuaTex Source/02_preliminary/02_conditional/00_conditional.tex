\subsection{Conditional Diffusion Models} \label{sec:conditional_diff}
In this section, the details are explained how the diffusion models can be conditioned. Conditioning a generative model means to control the output given some extra input. This input can be text describing an image \cite{rombach_high-resolution_2022}, or part on an image for inpainting \cite{rombach_high-resolution_2022} The extra input $\mathbf{c}$ is sometimes called the covariates but often times the conditional input. Diffusion models can be conditioned in two different ways. Firstly in section \ref{sec:conditional_diff_model}, during training, by inputting the condition $\mathbf{c}$ into the model and training accordingly \cite{ho_cascaded_2021}. Secondly in section \ref{sec:conditional_diff_guidance}, there is Diffusion Guidance, a novel approach that shifts the conditioning from training to the inference stage, leveraging the Bayes rule \cite{song_score-based_2021}. 

\subsubsection{Conditional Denoising Model} \label{sec:conditional_diff_model}
Conditional versions of the model can be extended to embrace additional input \( c \), altering the backward denoising step in \autoref{eq:diff_reverse_process} \cite{ho_cascaded_2021}:
\begin{equation} \label{eq:cond_reverse_process}
 p_\theta(x^{0:K} | \mathbf{c}) = p(x^K)\prod_{k=1}^{K} p_\theta(x^{k-1} \mid x^k, \mathbf{c}) \text{ with } p_\theta(x^{k-1}|x^k, \mathbf{c}) = \mathcal{N} (x^{k-1}; \mu_\theta(x^k, k|\mathbf{c}), \sigma^2_k \mathbf{I})   
\end{equation}
Consequently, the noise and data variants of the prediction models are conditioned as respectively:
\begin{equation} \label{eq:cond_ddpm_noise_prediction}
    \mu_\epsilon(\epsilon_\theta, \mathbf{c}) = \frac{1}{\sqrt{1 - \beta_k}}\left(x^k - \frac{\beta_k}{\sqrt{1 - \alpha_k}} \epsilon_\theta(x^k, k | \mathbf{c}) \right)
\end{equation}
\begin{equation} \label{eq:cond_ddpm_data_prediction}
    \mu_x(x_\theta, \mathbf{c}) = \frac{\sqrt{1 - \beta_k}(1 - \alpha_{k-1})}{1 - \alpha_k} x^k + \frac{\sqrt{\alpha_{k-1}}\beta_k}{1 - \alpha_k} x_\theta(x_k, k | \mathbf{c})
\end{equation}
where the conditioning in practice are extra inputs to the prediction model as illustrated in \autoref{fig:conditional_prediction_models}. The same condition is included in each backwards step.

\begin{figure}[ht] 
\centering
\begin{tikzpicture}[
    auto,
    >={Latex[width=2mm,length=2mm]},
    data/.style={rectangle, draw=black, fill=green!20, align=center, rounded corners, minimum width=1.5cm, minimum height=1cm},
    model/.style={rectangle, draw=black, fill=red!20, align=center, rounded corners, minimum width=2.5cm, minimum height=1cm},
    condition/.style={rectangle, draw=black, fill=yellow!50, align=center, rounded corners, minimum width=1.5cm, minimum height=1cm},
    every node/.style={font=\sffamily},
    node distance=1.5cm
]

% Nodes
\node (dots1) [right=1cm of x0] {\(\cdots\)};
\node[data] (xk-1) [right=1cm of dots1] {\(x^{k-1}\)}; 
\node[data] (xk) [right=9cm of xk-1] {\(x^k\)}; 
\node (dots2) [right=1cm of xk] {\(\cdots\)};

% Model node with the equations
\node[model] (nn) [below=25pt of $(xk-1)!0.5!(xk)$] {
    $x^{k-1} = \mu_\theta(x^k, k | \mathbf{c}) 
    \begin{cases}
            \mu_\epsilon(\epsilon_\theta, \mathbf{c}) & \eqref{eq:cond_ddpm_noise_prediction} \\
            \mu_x(x_\theta, \mathbf{c}) & \eqref{eq:cond_ddpm_data_prediction}
    \end{cases}
    + \sigma_k \mathbf{z}$
};

\node[condition] (cond) [right=of nn] {Condition: $\mathbf{c}$}; 

\node[above=3pt of nn] (equation_p) {$p_\theta(x^{k-1}|x^k, \mathbf{c}) = \mathcal{N} (x^{k-1}; \mu_\theta(x^k, k | \mathbf{c}), \sigma^2_k \mathbf{I})$};

% Forward process arrows
\draw[->] ([yshift=10pt] dots1.east) -- ([yshift=10pt] xk-1.west);
\draw[->] ([yshift=10pt] xk-1.east) -- ([yshift=10pt] xk.west); 
\draw[->] ([yshift=10pt] xk.east) -- ([yshift=10pt] dots2.west);

% Reverse process arrows
\draw[->] ([yshift=-10pt] dots2.west) --  ([yshift=-10pt] xk.east); 
\draw[->] ([yshift=-10pt] xk-1.west) -- ([yshift=-10pt] dots1.east); 

\draw[->] ([yshift=-10pt] xk.west) -- ([yshift=0pt] nn.east); 
\draw[->] ([yshift=0pt] nn.west) -- ([yshift=-10pt] xk-1.east);

\draw[->] (cond.west) -- (nn.east);

\end{tikzpicture}


\caption{Illustration for conditional prediction models of the reverse process.}
\label{fig:conditional_prediction_models}
\end{figure}

For the score-based generative models through SDE such conditioning also exists, where the conditioning happens in the marginal distributions of the score \cite{yan_scoregrad_2021}. The reverse-time SDE becomes: 
\begin{equation} \label{eq:cond_sde_reverse_time}
    dx = \left[ f(x, k) = g(k)^2 \nabla_{x} \log{p_k(x|\mathbf{c})} \right] dk + g(k)d\bar{w}
\end{equation}
and the objective function then becomes
\begin{equation} \label{eq:cond_sde_train_obj}
    \mathcal{L}_k = \mathbb{E}_{k, x(0), x(k)} \left[\left|\left| \nabla_{x(k)} \log{p_{0k}(x(k) | x(0))} - s_\theta(x(k),  k | \mathbf{c})\right|\right|^2 \right]    
\end{equation}.

\subsubsection{Diffusion Guidance} \label{sec:conditional_diff_guidance}
In traditional diffusion models, conditioning is typically introduced during the training process, as represented in equation \eqref{eq:cond_reverse_process}. However, a novel approach introduces conditions during the inference stage by employing the concept of guidance \cite{sohl-dickstein_deep_2015, song_score-based_2021}. This method is grounded in the application of the Bayes rule to the probability $p(x^k|\mathbf{c})$ given by:
\begin{align}
    p(x^k|\mathbf{c}) &= \frac{p(\mathbf{c}|x^k) \times p(x^k)}{p(\mathbf{c})} 
\end{align}
Taking the natural logarithm of both sides, we can further derive:
\begin{align} 
    \log{p(x^k|\mathbf{c})} &= \log{p(\mathbf{c}|x^k)} + \log{p(x^k)} - \log{p(\mathbf{c})} 
\end{align}
From which, upon differentiation with respect to $x^k$, we obtain:
\begin{align} \label{eq:guid_gradient_relation}
    \nabla_{x^k} \log p(x^k|\mathbf{c}) &= \nabla_{x^k} \log p(\mathbf{c}|x^k) + \nabla_{x^k} \log p(x^k) 
\end{align}
The relation in \eqref{eq:guid_gradient_relation} provides a bridge to guide the reverse diffusion process in equation \eqref{eq:diff_reverse_process} and \eqref{eq:sde_reverse_process}. Instead of merely reversing the diffusion, this guides the samples to conform to the desired condition $\mathbf{c}$ \cite{sohl-dickstein_deep_2015, song_score-based_2021}. Thus, equation \eqref{eq:diff_reverse_process} can be conditioned as:
\begin{equation} \label{eq:guid_reverse_process}
    p_\theta(x^{k-1} | x^k, \mathbf{c}) = \mathcal{N} (x^{k-1}; \mu_\theta(x^k, k), \sigma^2_k \mathbf{I}) + s\sigma^2_k\nabla_{x^k} \log p(\mathbf{c}|x^k) 
\end{equation}
Here, $s\sigma^2_k\nabla_{x^k} \log p(\mathbf{c}|x^k)$ is the guidance term that drives the generated image towards the target condition. $s$ is a scale parameter controlling the guidance strength. The gradient $\nabla_x \log{p_k(\mathbf{c} | x)}$ can be estimated in two ways: the first involves training an auxiliary model as described by \cite{luo_understanding_2022} and the second method, described by \textcite{song_score-based_2021, luo_understanding_2022}, which doesn't require a auxiliary model. In \autoref{fig:guidance_prediction_models} the $\nabla_{x^k} \log p(\mathbf{c}|x^k)$ is termed $g(\mathbf{c}, x^k)$ as the guidance gradient.

Such guidance can also be applied to the continuous diffusion models as described by \textcite{song_score-based_2021}. For equation \eqref{eq:sde_reverse_process}, it can be conditioned as:
\begin{equation}
    \md x = \left\{ f(x, k) - g(k)^2 \left[ \nabla_x \log{p_k(x)} + \nabla_x \log{p_k(\mathbf{c} | x)}  \right] \right\} \md k + g(k)d\bar{w}
\end{equation}

\begin{figure}[ht] 
\centering
\begin{tikzpicture}[
    auto,
    >={Latex[width=2mm,length=2mm]},
    data/.style={rectangle, draw=black, fill=green!20, align=center, rounded corners, minimum width=1.5cm, minimum height=1cm},
    model/.style={rectangle, draw=black, fill=red!20, align=center, rounded corners, minimum width=2.5cm, minimum height=1cm},
    condition/.style={rectangle, draw=black, fill=yellow!50, align=center, rounded corners, minimum width=1.5cm, minimum height=1cm},
    every node/.style={font=\sffamily},
    node distance=1.5cm
]

% Nodes
\node (dots1) [right=1cm of x0] {\(\cdots\)};
\node[data] (xk-1) [right=1cm of dots1] {\(x^{k-1}\)}; 
\node[data] (xk) [right=10cm of xk-1] {\(x^k\)}; 
\node (dots2) [right=1cm of xk] {\(\cdots\)};

% Model node with the equations
\node[model] (nn) [below=25pt of $(xk-1)!0.5!(xk)$] {
    $x^{k-1} = \mu_\theta(x^k, k) 
    \begin{cases}
            \mu_\epsilon(\epsilon_\theta) & \eqref{eq:cond_ddpm_noise_prediction} \\
            \mu_x(x_\theta) & \eqref{eq:cond_ddpm_data_prediction}
    \end{cases}
    + \sigma_k \mathbf{z} + s\sigma^2_k g(\mathbf{c}|x^k)$
};

\node[condition] (cond) [right=of nn] {Condition: $\mathbf{c}$};

\node[above=3pt of nn] (equation_p) {$p_\theta(x^{k-1}|x^k, \mathbf{c}) = \mathcal{N} (x^{k-1}; \mu_\theta(x^k, k), \sigma^2_k \mathbf{I}) + s\sigma^2_k g(\mathbf{c}|x^k)$};

% Forward process arrows
\draw[->] ([yshift=10pt] dots1.east) -- ([yshift=10pt] xk-1.west);
\draw[->] ([yshift=10pt] xk-1.east) -- ([yshift=10pt] xk.west); 
\draw[->] ([yshift=10pt] xk.east) -- ([yshift=10pt] dots2.west);

% Reverse process arrows
\draw[->] ([yshift=-10pt] dots2.west) --  ([yshift=-10pt] xk.east); 
\draw[->] ([yshift=-10pt] xk-1.west) -- ([yshift=-10pt] dots1.east); 

\draw[->] ([yshift=-10pt] xk.west) -- ([yshift=0pt] nn.east); 
\draw[->] ([yshift=0pt] nn.west) -- ([yshift=-10pt] xk-1.east);

\draw[->] (cond.west) -- (nn.east);

\end{tikzpicture}

\caption{Illustration for conditional prediction models of the reverse process.}
\label{fig:guidance_prediction_models}
\end{figure}

