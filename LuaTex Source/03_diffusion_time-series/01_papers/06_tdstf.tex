\subsection{TDSTF: Transformer-based Diffusion probabilistic model for Sparse Time series Forecasting (2023) \cite{chang_tdstf_2023}} \label{sec:tdstf}
The TDSTF model, drawing inspiration from CSDI \cite{tashiro_csdi_2021}, is specifically designed for forecasting ICU patient vital signs using sparse historical data. Addressing the challenge of sparse multivariate ICU data, the model introduces key innovations. Its main contribution is efficiently representing the spare data with a triplet form, which contains a feature, time, and value. On top of that, there is also a mask bit indicating the presence or absence of data of a triplet. This compact representation effectively stores sparse data, maintaining the integrity of temporal information and reducing the noise that typically arises from imputation or aggregation methods. Furthermore, as the name implies, a Transformer \cite{vaswani_attention_2017} is used to encode the historical data. 
There are also adjustments in selecting the inputs to the encoder. First of all, if the mask indicates the absence of a triplet, it won't be included to the Transformer. Furthermore, if there are too many triplets as input to the model, only the data points of the same target feature, or most correlated to the target data are included. This ensures that the most relevant information is included in the model's input.

The reverse process is formalized as in \autoref{eq:cond_time_reverse_process} with a noise prediction model and corresponding loss function 
\begin{equation}
    \mathcal{L}_{\epsilon_\theta} = \mathbb{E}_{k, X_{tar}^0, \epsilon} \left[|| \epsilon - \epsilon_\theta(X_{tar}^k, k | X_{obs})||^2 \right].
\end{equation}

The ICU data contains the vital signs Heart Rate (HR), Systolic Blood Pressure (SBP), and Diastolic Blood Pressure (DBP). Furthermore, it also contains special events that occur with the patient such as administering drugs. This approach allows for a more accurate forecast by considering the interrelated events, interventions, and conditions that could impact the patient.
The paper states that 60 triplets are used to encode all of the historical data, however it is not mentioned how many steps in the future the vital signs are predicted. The paper merely states it makes a 10-minute forecast of the three vital signs HR, SBP, and DBP.

The specific data set is MIMIC-III \cite{johnson_mimic-iii_2015}\lfootnote{tdstf}{TDSTF \cite{chang_tdstf_2023}:}{https://github.com/PingChang818/TDSTF} and the performance are evaluated with MSE and NACRPS as described in section \ref{sec:time-series_evaluation_metric}. The results are presented in \autoref{tab:tdstf-results}.

\begin{table}[ht]
    \centering
    \begin{tabular}{cccccccccc}
        \toprule
        \multirow{2}{*}{Dataset} & \multirow{2}{*}{H} & \multirow{2}{*}{F} & \multicolumn{2}{c}{NACRPS} & \multicolumn{2}{c}{MSE} \\
        \cmidrule(lr){4-5} \cmidrule(lr){6-7}
         & & & CSDI & TDSTF & CSDI & TDSTF \\
        \midrule
        \multirow{1}{*}{MIMIC-III} & 60tr  & 10m  & $0.5470 \pm 0.0040$ & \textbf{0.4438}$\pm$ \textbf{0.0078} & $0.6341 \pm 0.0098$ & \textbf{0.4168} $\pm$ \textbf{0.0232} \\
        \bottomrule
    \end{tabular}
    \caption{TDSTF ICU Vital Signs Forecasting results \cite{chang_tdstf_2023}}
    \label{tab:tdstf-results}
\end{table}

Besides achieving better predictive results, TSDTF is also much faster to train and inference compared to CSDI [\ref{sec:csdi}]. In the experiments, CSDI has 280 thousand parameters and TDSTF 560 thousand, even with this increase, TDSTF only took 30 minutes to train compared to the 12 hours for CSDI. Furthermore, CSDI took 14.913 seconds to inference while TDSTF only took 0.865 seconds, making TDSTF more than 17 times faster.
TSDTF only shown to be an improvement over CSDI for the sparse dataset MIMIC-III\footref{fn:tdstf}. More insights in its performance could be achieved by testing it against the same datasets mentioned in the CSDI paper \cite{tashiro_csdi_2021}. Overall the paper makes a strong case, solving the shortcomings of CSDI when it comes to sparse data.